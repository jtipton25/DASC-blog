\documentclass[11pt,]{article}
\usepackage[sc, osf]{mathpazo}
\usepackage{amssymb,amsmath}
\usepackage{ifxetex,ifluatex}
\usepackage{fixltx2e} % provides \textsubscript
\ifnum 0\ifxetex 1\fi\ifluatex 1\fi=0 % if pdftex
  \usepackage[T1]{fontenc}
  \usepackage[utf8]{inputenc}
\else % if luatex or xelatex
  \ifxetex
    \usepackage{mathspec}
  \else
    \usepackage{fontspec}
  \fi
  \defaultfontfeatures{Ligatures=TeX,Scale=MatchLowercase}
\fi
% use upquote if available, for straight quotes in verbatim environments
\IfFileExists{upquote.sty}{\usepackage{upquote}}{}
% use microtype if available
\IfFileExists{microtype.sty}{%
\usepackage{microtype}
\UseMicrotypeSet[protrusion]{basicmath} % disable protrusion for tt fonts
}{}
\usepackage[margin=1in]{geometry}




\setlength{\emergencystretch}{3em}  % prevent overfull lines
\providecommand{\tightlist}{%
  \setlength{\itemsep}{0pt}\setlength{\parskip}{0pt}}
\setcounter{secnumdepth}{0}
% Redefines (sub)paragraphs to behave more like sections
\ifx\paragraph\undefined\else
\let\oldparagraph\paragraph
\renewcommand{\paragraph}[1]{\oldparagraph{#1}\mbox{}}
\fi
\ifx\subparagraph\undefined\else
\let\oldsubparagraph\subparagraph
\renewcommand{\subparagraph}[1]{\oldsubparagraph{#1}\mbox{}}
\fi

% JRT added 2020-11-18
\usepackage{multicol}

% Now begins the stuff that I added.
% ----------------------------------

% Custom section fonts
\usepackage{sectsty}
\sectionfont{\rmfamily\mdseries\large\bf}
\subsectionfont{\rmfamily\mdseries\normalsize\scshape}


% Make lists without bullets
\renewenvironment{itemize}{
  \begin{list}{}{
    \setlength{\leftmargin}{1.5em}
  }
}{
  \end{list}
}


% Make parskips rather than indent with lists.
\usepackage{parskip}
\usepackage{titlesec}
\titlespacing\section{0pt}{12pt plus 4pt minus 2pt}{4pt plus 2pt minus 2pt}
\titlespacing\subsection{0pt}{12pt plus 4pt minus 2pt}{4pt plus 2pt minus 2pt}

% Use fontawesome. Note: you'll need TeXLive 2015. Update.
\usepackage{fontawesome}

% Fancyhdr, as I tend to do with these personal documents.
\usepackage{fancyhdr,lastpage}
\pagestyle{fancy}
\renewcommand{\headrulewidth}{0.0pt}
\renewcommand{\footrulewidth}{0.0pt}
\lhead{}
\chead{}
\rhead{}
\lfoot{
\cfoot{\scriptsize  John Tipton - CV }}
\rfoot{\scriptsize \thepage/{\hypersetup{linkcolor=black}\pageref{LastPage}}}

% Always load hyperref last.
\usepackage{hyperref}
\PassOptionsToPackage{usenames,dvipsnames}{color} % color is loaded by hyperref

\hypersetup{unicode=true,
            pdftitle={John Tipton:  CV (Curriculum Vitae)},
            pdfauthor={John Tipton},
            colorlinks=true,
            linkcolor=blue,
            citecolor=Blue,
            urlcolor=blue,
            breaklinks=true, bookmarks=true}
\urlstyle{same}  % don't use monospace font for urls

% Make AP style (kinda) dates for the updated/today field

\usepackage{datetime}
\newdateformat{apstylekinda}{%
  \shortmonthname[\THEMONTH]. \THEDAY, \THEYEAR}

\begin{document}


\centerline{\huge \bf John Tipton}

\vspace{2 mm}

\hrule

\vspace{2 mm}

\moveleft.5\hoffset\centerline{Assistant Professor, Department of Mathematical Sciences}
\moveleft.5\hoffset\centerline{309 Science and Engineering Building · University of Arkansas ·
Fayetteville, AR, 72701}
\moveleft.5\hoffset\centerline{ \faEnvelopeO \hspace{1 mm} \href{mailto:}{\tt \href{mailto:jrtipton@uark.edu}{\nolinkurl{jrtipton@uark.edu}}} \hspace{1 mm}      \faGlobe \hspace{1 mm} \href{http://www.jrtipton.com}{\tt www.jrtipton.com}    | \emph{Updated:} \apstylekinda\today} 

\vspace{2 mm}

\hrule


\hypertarget{employment}{%
\section{Employment}\label{employment}}

\hypertarget{university-of-arkansas-department-of-mathematical-sciences}{%
\subsection{\texorpdfstring{University of Arkansas \hfill Department of
Mathematical
Sciences}{University of Arkansas Department of Mathematical Sciences}}\label{university-of-arkansas-department-of-mathematical-sciences}}

Assistant Professor \hfill 2017--present

\hypertarget{colorado-state-university-department-of-statistics}{%
\subsection{\texorpdfstring{Colorado State University \hfill Department
of
Statistics}{Colorado State University Department of Statistics}}\label{colorado-state-university-department-of-statistics}}

Postdoctoral Researcher \hfill 2016--2017\\
Graduate Research Assistant \hfill 2009-2016

\hypertarget{education}{%
\section{Education}\label{education}}

\hypertarget{colorado-state-university-ph.d.-statistics-2016}{%
\subsection{\texorpdfstring{Colorado State University, Ph.D.~Statistics
\hfill 2016}{Colorado State University, Ph.D.~Statistics 2016}}\label{colorado-state-university-ph.d.-statistics-2016}}

\hypertarget{colorado-state-university-m.s.-statistics-2011}{%
\subsection{\texorpdfstring{Colorado State University, M.S. Statistics
\hfill 2011}{Colorado State University, M.S. Statistics 2011}}\label{colorado-state-university-m.s.-statistics-2011}}

\hypertarget{colorado-state-university-b.s.-mathematics-and-zoology-2005}{%
\subsection{\texorpdfstring{Colorado State University, B.S. Mathematics
and Zoology
\hfill 2005}{Colorado State University, B.S. Mathematics and Zoology 2005}}\label{colorado-state-university-b.s.-mathematics-and-zoology-2005}}

\hypertarget{assignment}{%
\section{Assignment}\label{assignment}}

Assigned to 40\% teaching, 40\% research, and 20\% service. My full
teaching load is 2 courses per semester including development of new
course materials. My research is focused on applied statistics and
methodological statistics research with an emphasis on spatial,
paleoclimate, and environmental applications.

\hypertarget{publications}{%
\section{Publications}\label{publications}}

\hypertarget{published-and-in-review}{%
\subsection{Published and in Review}\label{published-and-in-review}}

\begin{itemize}
\item
  {[}15{]} John Tipton, Glenn Sharman, and Sam Johnstone (2020+).
  Estimation of Sediment Mixing Distributions with Uncertainty
  Quantification using Bayesian Nonparametric Modeling. In revision.
  \href{https://github.com/jtipton25/mixing-manuscript}{{[}R code on
  gitHub{]}}
\item
  {[}14{]} Ann Raiho, Michael Dietze, Andria Dawson, Christine R.
  Rollinson, John Tipton, and Jason McLachlan. (2020+). Towards
  understanding predictability in ecology: A forest gap model case
  study. Global Change Biology. In review.
  \href{https://www.biorxiv.org/content/10.1101/2020.05.05.079871v2.full.pdf}{{[}bioRxiv{]}}
\item
  {[}13{]} Manuel Chevalier et. al (2020). Pollen-based climate
  reconstruction techniques for late Quaternary studies. Earth-Science
  Reviews.
  \href{https://doi.org/10.1016/j.earscirev.2020.103384}{{[}paper{]}}
\item
  {[}12{]} John Tipton, Mevin Hooten, Connor Nolan, Bob Booth, and Jason
  McLachlan (2019). Predicting unobserved climate from compositional
  data using multivariate Gaussian process inverse prediction. The
  Annals of Applied Statistics. Vol 13, No.~4, pp.~2363-2388.
  \href{https://projecteuclid.org/euclid.aoas/1574910048}{{[}paper{]}}\href{https://arxiv.org/abs/1903.05036}{{[}arXiv
  preprint{]}}{[}\href{https://github.com/jtipton25/compositional-inverse-prediction}{R
  code on gitHub}
\item
  {[}11{]} Connor Nolan, John Tipton, Robert K. Booth, Mevin B. Hooten,
  and Stephen T. Jackson (2019). Comparing and improving methods for
  reconstructing water table depth from testate amoebae. The Holocene.
  \href{https://journals.sagepub.com/doi/10.1177/0959683619846969}{{[}paper{]}}\href{https://github.com/jtipton25/BayesComposition}{{[}R
  code on gitHub{]}}
\item
  {[}10{]} Connor Nolan et. al.~(2018) Past and future global
  transformation of terrestrial ecosystems under climate change.
  Science. Volume 361, Issue 6405, August 31, pp.~920-923.
  \href{http://science.sciencemag.org/content/361/6405/920}{{[}paper{]}}
\item
  {[}9{]} Erin Belval, Yu Wei, David Calkin, Crystal Stonesifer, Matt
  Thompson and John Tipton (2017). Studying interregional wildland
  engine assignments for large fire suppression. International Journal
  of Wildland Fire.
  \href{http://www.publish.csiro.au/WF/WF16162}{{[}paper{]}}
\item
  {[}8{]} John Tipton, Mevin Hooten, and Simon Goring (2017).
  Reconstruction of spatio-temporal temperature processes from sparse
  historical records using probabilistic principal component regression.
  Advances in Statistical Climatology, Meteorology and Oceanography.
  \href{http://www.adv-stat-clim-meteorol-oceanogr.net/3/1/2017/ascmo-3-1-2017.pdf}{{[}pdf{]}}
  \href{https://github.com/jtipton25/observer}{{[}R code on gitHub{]}}
\item
  {[}7{]} Trevor Hefley, Kristin Broms, Brian Brost, Frances Buderman,
  Shannon Kay, Henry Scharf, John Tipton, Perry Williams, and Mevin
  Hooten (2017). The basis function approach to modeling dependent
  ecological data. Ecology.
  \href{http://www.stat.colostate.edu/~hooten/papers/pdf/Hefley_etal_Ecology_2017.pdf}{{[}pdf{]}}
  {[}R code
  \href{http://onlinelibrary.wiley.com/store/10.1002/ecy.1674/asset/supinfo/ecy1674-sup-0003-AppendixS3.pdf?v=1\&s=1ec1d5c25a2287b36029dd315184190d0c5f8c04}{S3},
  \href{http://onlinelibrary.wiley.com/store/10.1002/ecy.1674/asset/supinfo/ecy1674-sup-0004-AppendixS4.pdf?v=1\&s=4c29dbe01f5521638315a3159c43971b30822d0c}{S4},
  \&
  \href{http://onlinelibrary.wiley.com/store/10.1002/ecy.1674/asset/supinfo/ecy1674-sup-0005-AppendixS5.pdf?v=1\&s=8bbe6b6f8165283cf91df19b528a34b8ee1ee6d4}{S5}{]}
\item
  {[}6{]} John Tipton, Mevin Hooten, Neil Pederson, Martin Tingley and
  Daniel Bishop (2016). Reconstruction of late Holocene climate based on
  tree growth and mechanistic hierarchical models. Environmetrics,
  27(1):42-54.
  \href{http://onlinelibrary.wiley.com/doi/10.1002/env.2368/abstract}{{[}pdf{]}}
  \href{https://github.com/jtipton25/Mechanistic-Tree-Ring}{{[}R code on
  gitHub{]}}
\item
  {[}5{]} Douglas Silver, Brett Johnson, William Pate, Kyle
  Christianson, John Tipton, James Sherwood, Brian Smith, Yun Hao, and
  Patrick Martinez (2016). Effect of net size on estimates of abundance,
  size/age, and sex of Mysis diluviana. Journal of Great Lakes Research,
  Volume 42, Issue 3, June 2016
  \href{http://www.sciencedirect.com/science/article/pii/S038013301600040X?via\%3Dihub}{{[}paper{]}}
  \href{http://jtipton25.github.io/mysis/}{{[}R code on gitHub{]}}
\item
  {[}4{]} John Tipton, Jean Opsomer, and Gretchen Moisen (2013).
  Properties of endogenous post-stratified estimation using remote
  sensing data. Remote Sensing of Environment, 139:130-137.
  \href{https://www.fs.fed.us/rm/pubs_other/rmrs_2013_tipton_j001.pdf}{{[}pdf{]}}
\item
  {[}3{]} John Tipton, Gretchen Moisen, Paul Patterson, and Thomas
  Jackson (2012). Sampling intensity and normalization: Exploring
  cost-driving factors in nationwide mapping of tree canopy cover. In:
  McWilliams, Will and Roesch, Frank, (compilers). 2010 Forest Inventory
  and Analysis (FIA) Symposium.
  \href{https://www.srs.fs.fed.us/pubs/gtr/gtr_srs157/gtr_srs157_201.pdf}{{[}pdf{]}}
\item
  {[}2{]} Thomas Jackson, Gretchen Moisen, Paul Patterson and John
  Tipton (2012). Repeatability in photo-interpretation of tree canopy
  cover and its effect on predictive mapping. In:McWilliams, Will and
  Roesch, Frank, (compilers). 2010 Forest Inventory and Analysis (FIA)
  Symposium.
  \href{https://www.srs.fs.fed.us/pubs/gtr/gtr_srs157/gtr_srs157_189.pdf}{{[}pdf{]}}
\item
  {[}1{]} David Gammon, Myron Baker, and John Tipton (2005). Cultural
  divergence within novel song in the black-capped chickadee (Poecile
  atricapillus). The Auk, 122, 853-871.
  \href{http://www.bioone.org/doi/abs/10.1642/0004-8038\%282005\%29122\%5B0853\%3ACDWNSI\%5D2.0.CO\%3B2}{{[}paper{]}}
\end{itemize}

\hypertarget{in-preparation}{%
\subsection{In Preparation}\label{in-preparation}}

\begin{itemize}
\item
  John Tipton, Basil Davis, Manuel Chevalier, and Philipp Sommer.
  Spatio-temporal reconstruction of climate from pollen.
\item
  Kelly Heilman, Jason McLachlan, A. Carla Staver, John W.Williams,
  David Mladenoff, Simon Goring, Christopher J. Paciorek, and John
  Tipton. Loss of temperate savanna-forest bistability due to changes in
  land-use.
\item
  John Tipton, Basil Davis, Manuel Chevalir, and Philipp Sommer.
  Improving pollen reconstructions of climate using Polya-gamma data
  augmentation.
\item
  John Tipton. Making Bayesian spatio-temporal models conjugate through
  recursive Bayesian inference.
\end{itemize}

\hypertarget{presentations}{%
\section{Presentations}\label{presentations}}

\hypertarget{invited}{%
\subsection{Invited}\label{invited}}

\begin{itemize}
\item
  2019 -- November -- Presentation - University of Arkansas Industrial
  Engineering INFORMS Seminar: Reducing the computational cost for
  Bayesian modeling of non-Gaussian, noisy spatio-temporal data
\item
  2019 -- July -- Poster - Institute of Mathematical Statistics New
  Researchers Conference: Spatio-temporal reconstruction of climate from
  pollen data
\item
  2019 -- April -- Presentation - University of Missouri Statistics
  Department: Spatio-temporal reconstruction of climate from pollen data
\item
  2018 -- November -- Presentation - University of Arkansas Department
  of Geosciences: Don't let your statistics ruin the science that you
  love
\item
  2018 -- September -- Presentation - Kansas State University Statistics
  Department: Spatio-temporal reconstruction of climate from pollen data
\item
  2018 -- September -- Presentation - University of Arkansas Statistics
  Seminar: Spatio-temporal reconstruction of climate from pollen data
\item
  2017 -- August -- Poster - ASA: Bayesian Multispecies Ecological
  Models for Paleoclimate Reconstruction Using Inverse Prediction
\item
  2015 -- August -- Presentation - ASA: ENVR Student Paper Award:
  Reconstruction of late Holocene climate based on tree growth and
  mechanistic hierarchical models
\item
  2015 -- August -- Poster - ASA STATMOS: Reconstruction of late
  Holocene climate based on tree growth and mechanistic hierarchical
  models
\item
  2014 -- December -- Presentation - American Geophysical Union: A
  statistical reconstruction of bivariate climate from tree ring
  measurements using scientifically motivated process models
\end{itemize}

\hypertarget{contributed}{%
\subsection{Contributed}\label{contributed}}

\begin{itemize}
\item
  2019 -- August -- Presentation - ASA: Spatio-temporal reconstruction
  of climate from pollen data
\item
  2018 -- August -- Presentation - ASA: Modeling Sediment Mixing using
  Mixtures of Dirichlet Processes
\item
  2017 -- August -- Presentation - ASA: Reconstruction of
  Spatio-Temporal Temperature from Sparse Historical Records Using
  Robust Probabilistic Principal Component Regression
\item
  2017 -- December -- Poster: American Geophysical Union - co-author:
  Comparing and improving reconstruction methods for proxies based on
  compositional data
\item
  2017 -- December -- Presentation: American Geophysical Union -
  co-author: Inferring biogeochemistry past: a millennial-scale
  multimodel assimilation of multiple paleoecological proxies
\item
  2017 -- December -- Presentation: American Geophysical Union -
  co-author: Uncertainty and inference in the world of paleoecological
  data
\item
  2016 -- August -- Presentation - ASA: Inverting the Gaussian process:
  A Bayesian multispecies ecological model for paleoclimate
  reconstruction
\item
  2016 -- August -- Poster - ASA: Robust spatio-temporal reconstruction
  of temperature processes from sparse historical data
\item
  2014 -- December -- Poster - American Geophysical Union - co-author:
  Effects of European land use on contemporary tree-climate
  relationships in the northeastern United States: Implications for
  predictive models
\item
  2014 -- August -- Presentation - ASA: Reconstruction of historical
  climate using a reduced rank predictive process model
\item
  2012 -- August -- Presentation - ASA: Endogenous post-stratification
  using random forests
\item
  2012 -- December -- Presentation - Forest Inventory and Analysis
  Symposium: Properties of the endogenous post-stratified estimator
  using a random forest model
\end{itemize}

\hypertarget{funding}{%
\section{Funding}\label{funding}}

\hypertarget{external}{%
\subsection{External}\label{external}}

\begin{itemize}
\tightlist
\item
  2019-2024 -- National Science Foundation. Multidisciplinary Data
  Science Education to Prepare STEM Students for Data Science Careers
  Award: DUE 1930532 \$1,000,000 (role: Senior Personnel)
\end{itemize}

\hypertarget{not-funded}{%
\subsubsection{Not Funded}\label{not-funded}}

\hypertarget{internal}{%
\subsection{Internal}\label{internal}}

\begin{itemize}
\tightlist
\item
  2019-2022 -- University of Arkansas Chancellor's Grant. Computational
  Capsule Neural Network Algorithms for Enabling Terahertz Detection of
  Breast Cancer. \$53,197 (role: co-PI)
\end{itemize}

\hypertarget{teaching-experience}{%
\section{Teaching Experience}\label{teaching-experience}}

\hypertarget{university-of-arkansas}{%
\subsection{University of Arkansas}\label{university-of-arkansas}}

Responsible for course redesign and/or development for DASC 1104, DASC
2594, STAT 3003, STAT 4003, STAT 4043, STAT 5003 STAT 5383, and STAT
5413. Also a member of the data science curriculum development committee
developing 54 new credit hours and a member of the data science advisory
board. As a member of the data science curriculum committee, I created
lesson and unit plans for DASC 1104 Programming Languages for Data
Sciene, DASC 2594 Multivariable Math for Data Scientists, and DASC 3213
Statistical Learning

\begin{itemize}
\tightlist
\item
  DASC 1104 Programming Languages for Data Science (Fall 2020)
\item
  DASC 2594 Multivariable Math for Data Scientists (Spring 2021)
\item
  STAT 2303 Principles of Statistics (Fall 2017)
\item
  STAT 3003 Statistical Methods (Spring 2021)
\item
  STAT 5383 Time Series (Fall 2017)
\item
  STAT 4043 Survey Sampling (Spring 2018, Spring 2019)
\item
  STAT 4003 Statistical Methods (Fall 2018, Spring 2019, Fall 2019)
\item
  STAT 5003 Statistical Methods (Spring 2020, Fall 2020)
\item
  STAT 5413 Spatial Statistics (Spring 2020)
\end{itemize}

\hypertarget{colorado-state-university}{%
\subsection{Colorado State University}\label{colorado-state-university}}

\hypertarget{instructor}{%
\subsubsection{Instructor}\label{instructor}}

\begin{itemize}
\tightlist
\item
  STAT 307 Intro to Biostatistics, three semesters
\item
  STAT 204 Business Statistics, two semesters
\end{itemize}

\hypertarget{teaching-assistant}{%
\subsubsection{Teaching assistant}\label{teaching-assistant}}

\begin{itemize}
\tightlist
\item
  STAT 472 Statistical Consulting, one semester
\item
  STAT 301 Intro to Statistics Online, one semester
\item
  STAT 204 Business Statistics, one semester
\end{itemize}

\hypertarget{workshops}{%
\subsubsection{Workshops}\label{workshops}}

\begin{itemize}
\tightlist
\item
  Building Capacity in Bayesian Modeling for Ecologists (NSF), 2 days
  \hfill 2014, 2016
\item
  Workshop on Parallel Computing, CU/CSU, \hfill 2015
\item
  R Workshop (CSU-CCFRWU), 1 day \hfill 2013, 2015
\end{itemize}

\hypertarget{service}{%
\section{Service}\label{service}}

\begin{itemize}
\tightlist
\item
  2017-2020: Interdisciplinary committee for the creation of an
  undergraduate degree in data science
\item
  2019-present: Department of Mathematical Sciences undergraduate
  curriculum committee
\item
  2019-present: Data science undergraduate curriculum committee
\end{itemize}

\hypertarget{honorsawards}{%
\section{Honors/Awards}\label{honorsawards}}

\begin{itemize}
\tightlist
\item
  University of Arkansas New Faculty Commendation for Teaching
  Commitment \hfill 2018
\item
  Colorado State University Statistics Department Poster Symposium -
  Best Poster \hfill 2016
\item
  American Statistical Association ENVR Student Paper Competition Award
  \hfill 2015
\item
  Thomas J. and Eileen C. Boardman Statistical Consulting Award
  \hfill 2014
\item
  American Statistical Association Student Travel Award \hfill 2014
\item
  American Statistical Association Wray Jackson Smith Award \hfill 2012
\end{itemize}

\hypertarget{software-and-computing}{%
\section{Software and Computing}\label{software-and-computing}}

\hypertarget{r-packages}{%
\subsection{R packages}\label{r-packages}}

\begin{itemize}
\tightlist
\item
  \href{https://github.com/jtipton25/BayesMRA}{R package BayesMRA --
  available on CRAN} \hfill 2020
\item
  \href{https://github.com/jtipton25/BayesComposition}{gitHub R package
  Bayescomposition} \hfill 2019
\item
  \href{https://github.com/jtipton25/stPollen}{gitHub R package
  stPollen} \hfill 2019
\end{itemize}

\hypertarget{data-tutorials-and-reproducibility}{%
\subsection{Data Tutorials and
Reproducibility}\label{data-tutorials-and-reproducibility}}

\begin{itemize}
\item
  \href{https://github.com/jtipton25/compositional-inverse-prediction/}{GitHub
  tutorial} for: John Tipton, Mevin Hooten, Connor Nolan, Bob Booth, and
  Jason McLachlan. (2019) Predicting unobserved climate from
  compositional data using multivariate Gaussian process inverse
  prediction. \textit{Annals of Applied Statistics}
\item
  \href{https://jtipton25.github.io/mysis/}{GitHub tutorial} for:
  Douglas Silver, Brett Johnson, William Pate, Kyle Christianson, John
  Tipton, James Sherwood, Brian Smith, Yun Hao, and Patrick Martinez.
  (2016). Effect of net size on estimates of abundance, size/age, and
  sex of \textit{Mysis diluviana}.
  \textit{Journal of Great Lakes Research}, Volume 42, Issue 3.
\item
  \href{https://github.com/jtipton25/Mechanistic-Tree-Ring}{GitHub
  tutorial and R software} for: John Tipton, Mevin Hooten, Neil
  Pederson, Martin Tingley and Daniel Bishop. (2016). Reconstruction of
  late Holocene climate based on tree growth and mechanistic
  hierarchical models. \textit{Environmetrics}, 27(1):42-54.
\end{itemize}

\hypertarget{computing-expertise}{%
\subsection{Computing Expertise}\label{computing-expertise}}

\begin{multicols}{3}
\begin{itemize}
\item \textsc{R} 
\item \textsc{C++}
\item \LaTeX 
\item \textsc{Linux}
\item \textsc{python}
\end{itemize}
\end{multicols}

\hypertarget{student-advising}{%
\section{Student Advising}\label{student-advising}}

\hypertarget{graduate-students-current}{%
\subsection{Graduate Students
(Current)}\label{graduate-students-current}}

\begin{itemize}
\tightlist
\item
  Jean Remy Habima (PhD-Mathematics), Adviser
\item
  Surya Lamichhane (PhD-Mathematics), Committee Member
\item
  Muhenned Abdulasalim (MS-STAN), Adviser
\item
  Thuy Scanlon (MS-STAN), Adviser
\item
  Ariel Mundo (PhD-Biomedical Engineer), Committee Member
\item
  Leah Bayer (PhD-Biology), Committee Member
\item
  Chang Liu (PhD-Environmental Dynamics), Proposal Committee Member
\end{itemize}

\hypertarget{graduate-students-graduated}{%
\subsection{Graduate Students
(Graduated)}\label{graduate-students-graduated}}

\begin{itemize}
\tightlist
\item
  Pauline Morin (MS-Food Science), Committee Member \hfill 2020
\item
  Seyed Tabari (MS-STAN), Committee Member \hfill 2020
\item
  Philipp Sommer (PhD-Geosciences, University of Lausanne),
  \textbf{Examiner} \hfill 2019
\item
  Ruizhe (Rachel) Yin (MS-STAN), \textbf{Committee Chair} \hfill 2019
\item
  Hua Zhong (MS-STAN), \textbf{Committee Chair} \hfill 2019
\item
  Michael Harris (MS-STAN), Committee Member \hfill 2019
\item
  Josh Price (MS-STAN), Committee Member \hfill 2019
\item
  Md Kamrul Hasan Khan (MS-STAN), Committee Member \hfill 2018
\item
  Kai Cui (MS-STAN), Committee Member \hfill 2018
\item
  Michael Ellis (MS-STAN), Committee Member \hfill 2018
\end{itemize}

\hypertarget{undergraduate-students}{%
\subsection{Undergraduate Students}\label{undergraduate-students}}

\begin{itemize}
\tightlist
\item
  Abigail Rhodes (Honors)
\item
  Caleigh Christensen (Honors)
\item
  Brenna Frandson (Honors)
\end{itemize}

\hypertarget{manuscript-peer-review}{%
\section{Manuscript Peer Review}\label{manuscript-peer-review}}

Annals of Applied Statistics (4 times); Computational Statistics and
Data Analysis (2 times); Environmental and Ecological Statistics (1
time); Journal of Agricultural, Biological, and Environmental Statistics
(3 times); Journal of Geophysical Research - Biogeosciences (1 time);
Ecological Applications (1 time); Freshwater Science (1 time); Climate
of the Past (1 time); Chemical Geology (1 time); Arctic, Antarctic, and
Alpine Research (1 time); Science (1 time); Nature Communications (1
time)

\end{document}
